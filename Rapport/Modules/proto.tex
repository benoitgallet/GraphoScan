Pour bien prendre en main le projet il nous fallait tester réellement le dispositif de lancement du logiciel jusqu'à la capture vidéo. Nous avons du  procéder à l'installation de tout l'environnement de travail nécessaire (FlyCap 2, OpenCV, OpenGl) et l’acquisition des premières vidéos avec les caméras mises à notre disposition. \\

Le groupe affecté à la capture des vidéos en stéréo a pu transférer le code initial sous Linux et ainsi le tester directement. Les tests ont été concluants et ils ont pu commencer directement à améliorer le système. Le second groupe a quant à lui récupéré le code concernant le tracking mais a rencontré de gros problèmes lors de son passage de Windows vers Linux. \\
Le code n'utilisant pas des fonctions standards, important des librairies en "dur" et étant peu commenté, a rendu pour le moment impossible les tests du code de l'étudiant qui travaillait sur le tracking. Pour y remédier ils ont du repasser cette partie du projet sur Windows le temps de bien comprendre les différents problèmes et de les corriger.