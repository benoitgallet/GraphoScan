\subsection{Présentation}

Ce projet de TER prolonge un travail déjà entamé l'année dernière par deux étudiants de Polytech, consistant à enregistrer en vidéo l'écriture d'un calligraphe, pour pouvoir reconstruire un modèle en 3D du mouvement de la plume. Une structure en bois supporte deux caméras, que l'on peut bouger le long de rails puis fixer à l'aide de vis. Le calligraphe écrit sous cette structure et la feuille est éclairée par des spots lumineux. Il faut alors associer l'image des deux caméras, ce qui n'est pas possible nativement avec le logiciel fourni par le fabricant (FlyCapture de PointGrey) pour faire de l'acquisition vidéo en stéréo, puis reconstituer via OpenGL les mouvements de la plume. Ces mouvements ont été sauvegardés grâce à des algorithmes de tracking, travaillant sur les vidéos enregistrées auparavant.



\subsection{But du projet}

Ce projet avait pour objectif de réaliser une reconstitution 3D des mouvements du calligraphe. Celui-ci peut recopier plusieurs textes, provenant de différents lieux et différentes époques, afin de pouvoir comparer les styles d'écriture, définir s'il existait différentes écoles d'écriture, différents styles, etc. De manière plus générale, le projet pourra servir pour beaucoup d'applications par la suite, car le code final se veut le plus généraliste possible.

La plupart des travaux actuels sur l'écriture utilisent des tablettes tactiles ou des capteurs placés à l'intérieur des stylos. Mais comme dans notre cas on des outils d'écriture spécifiques, une feuille et une plume ou un calame (morceau de roseau taillé en pointe), la captation du mouvement ne peut se faire a priori qu'indirectement, par le biais de capture de vidéos et de traitement d'images.